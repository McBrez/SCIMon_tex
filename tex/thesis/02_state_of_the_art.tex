\chapter{State of the Art}

\section{Micorfluidic Structures}
\textcolor{red}{This section explains microfluidic structures}

Curto et al. \cite{Curto2018} proposed a microfluidic system, that enables in-vitro \gls{is}. An \gls{oect} is facilitated as detector and amplifier for the measurement signal. The publication defines the spheroid resistance $R_{sph}$, which is expected to better describe the organoid than \gls{teer}. The organoid is held in place within an nozzle trap. This allows precise positioning and unloading/loading via an inversion of the medium flow direction. However, no continuous flow can be upheld, as this could cause the destruction of the organoid inside the nozzle trap. The authors suggest that this setup is eligible to distinguish between cell types an their viability. However, almost no effect has been reported on variations of spheroid size. 

Gong et al. \cite{Gong2021} followed a simpler approach. The \gls{is} is conveyed through a field's metal electrode structure. The spheroid is not held in place. Rather it is pumped through a closed loop system and is analyzed, every time it passes the electrode array. This setup lends itself to high throughput applications, but is less suitable for configurations where single spheroids shall be monitored continuously.

Honrad et al. \cite{Honrado2021} show that \gls{is} is even feasible for analytes that comprise of a single cell. The presented microfluidics device includes multiple traps and bypass channels. This construction allows the simultaneous cytometry of multiple cells. 

\section{Analytical Algorithms}
\textcolor{red}{This section identifies algorithms that are used for analytics on spheroids.}
