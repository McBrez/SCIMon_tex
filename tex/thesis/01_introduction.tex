\chapter{Introduction}
Conventional in-vitro cell monitoring is mainly done on 2D cultures (i.e. bio films) \cite{Cho2020,Ranga2014}. Such sensing setups are quite simple and often comprise of nothing more than an semi-permeable membrane with deposited metal electrodes \cite{Hickman2016}. Although easy to handle, such measurements are not representative, as 2D cultures cannot depict the processes of in-vivo material. This circumstance can be traced back to missing cell-to-cell interactions. Recent studies show, that 3D cultures like spheroids or organoids resemble in-vivo cell behavior much more closely than their two-dimensional counterpart \cite{Ranga2014, Elliott2011}. Complementary to that, novel commercial products\cite{insphero, matrigel} allow the efficient and reliable growth of such cultures. Although a lot of research is focused on reactions of spheroid and organoids to medicinal compounds, there seems to be a lack of technical devices that facilitate the production of evidence \cite{Curto2018}. 
In order to address this shortfall, this work shall produce a system that allows the automatic and continuous monitoring of spheroids via an \gls{is} measurement. Usage of this system shall dramatically decrease the effort of cytotoxicity assays and shall pave the way for large scale clinical applications. With the ultimate goal being the acceleration of cancer/drug research, therapy and diagnostics. 