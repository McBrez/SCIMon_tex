\usepackage{cite}
\usepackage{pgfgantt}
\usepackage{pdflscape}
\usepackage{geometry}
\usepackage{glossaries}
\usepackage{cleveref}

\newcounter{myWeekNum}
\stepcounter{myWeekNum}

\newcommand{\myWeek}{\themyWeekNum
    \stepcounter{myWeekNum}
    \ifnum\themyWeekNum=53
    \setcounter{myWeekNum}{1}
    \else\fi
}

\makeglossaries
\newglossaryentry{cvd}
{
    name={CVD},
    first={chemical vapor depostion (CVD)},
    description={Chemical vapor deposition; A method for the growing of thin films that relies on the reaction of precursor gases on the target substrate}
}

\newglossaryentry{mbe}
{
    name={MBE},
    first={molecular beam epitaxy (MBE)},
    description={Molecular beam epitaxy; A method for the growing of thin films that relies on the evaporization of the target material in high vacuum and subsequent desublimation on the target substrate}
}

\newglossaryentry{tmd}
{
    name={TMD},
    plural={TMDs},
    firstplural={transition metal dichalcogenides (TMDs)},
    first={transition metal dichalcogenide (TMD)},
    description={A semiconductor of the type $MX_{2}$. Where M is a transition metal and X is a Calcogenide}
}

\newglossaryentry{cmos}
{
    name={CMOS},
    first={complementary metal oxide semiconductor (CMOS)},
    description={Complementary metal oxide semiconductor; Describes an integrated semiconductor component or the process of fabricating it}
}

\newglossaryentry{uv}
{
    name={UV},
    first={ultraviolett (UV)},
    description={Ultraviolet; Light with a wavelength of 100 nm to 315 nm}
}

\newglossaryentry{ald}
{
    name={ALD},
    first={atomic layer deposition (ALD)},
    description={Atomic layer deposition; A fabrication method, belonging to the \gls{cvd} category, which allows deposition of single atom layers}
}

\newglossaryentry{dep}
{
    name={DEP},
    first={dielectrophoresis (DEP)},
    description={Dielectrophoresis; A fabrication method, whereby particles are deposited with the help of non-uniform electric fields}
}

\newglossaryentry{afm}
{
    name={AFM},
    first={atomic force microscopy (AFM)},
    description={Atomic force microscopy; An imaging method which allows the resolution of single atoms.}
}

\newglossaryentry{eds}
{
    name = {EDS},
    first={energy-dispersive X-ray spectroscopy},
    description={Energy-dispersive X-ray spectroscopy (EDS); An analytical method for used for the chemical characterization of a sample}
}

\newglossaryentry{soi}
{
    name={SOI},
    first={silicon on insulator (SOI)},
    description={Silicon on insulator; Type of fabrication method that relies on wafers with buried oxide layers surrounded by layers of silicon}
}

\newglossaryentry{ipa}
{
    name={IPA},
    first={isopropyl alcohol (IPA)},
    description={Isopropyl alcohol; A type of solvent, usually used for cleaning purposes}
}

\newglossaryentry{ndir}
{
    name={NDIR},
    first={nondispersive infrared (NDIR)},
    description={Nondispersive infrared; Refers to a type of spectrometer which does not contain prisms. Is often used as gas sensor}
}

\newglossaryentry{tft}
{
    name={TFT},
    first={thin film transitor (TFT)},
    description={Thin film transistor; A special type of transistor that is often grown on non-semiconducting substrate (e.g. glass))}
}

\newglossaryentry{tem}
{
    name={TEM},
    first={transition electron microscopy (TEM)},
    description={Transition electron microscopy; An imaging method commonly used in nanotechnology}
}

\newglossaryentry{mosfet}
{
    name={MOSFET},
    first={metal oxide field effect transistor (MOSFET)},
    description={Metal oxide field effect transistor; One of the most common transistor types}
}

\newglossaryentry{beol}
{
    name={BEOL},
    first={back-end-of-line (BEOL)},
    description={Back-end-of-line; Part of the die fabrication, where the individual devices get connected}
}


\newglossaryentry{oect}
{
    name={OECT},
    first={organic electrochemical transistor (OECT)},
    description={DEFINE ME}
}

\newglossaryentry{teer}
{
    name={TEER},
    first={transI(epi/endo)-thelial resistance (TEER)},
    description={transI(epi/endo)-thelial resistance; A metric that describes the cell adhesion, proliferation or differentiation}
}

\newglossaryentry{is}
{
    name={IS},
    first={impedance spectroscopy (IS)},
    description={Impedance spectroscopy; A characterisation technique, where the impedance of the analyte under varying frequency is analyzed}
}

\newglossaryentry{gui}
{
    name={GUI},
    first={graphical user interface (GUI)},
    description={DEFINE ME}
}

\begin{document}

\setcounter{myWeekNum}{39}
\title{Research Proposal: Development of an automated microfluidic system for spheroid-on-chip impedance monitoring.}
\author{David Freismuth}

\maketitle

\abstract{A microfluidic setup comprising existing commercial microfluidic pumps, impedance analyzers, and customized chips will be realized for automated, continuous, real-time impedance based monitoring of cancer spheroids. This setup will reveal insights to in-vitro growth and proliferation of 3D cancer models. It will also allow to study the influence of conventional and novel anti-cancer drugs.
}
 
\section{Introduction}
Conventional in-vitro cell monitoring is mainly done on 2D cultures (i.e. bio films) \cite{Cho2020,Ranga2014}. Such sensing setups are quite simple and often comprise of nothing more than an semi-permeable membrane with deposited metal electrodes \cite{Hickman2016}. Although easy to handle, such measurements are not representative, as 2D cultures cannot depict the processes of in-vivo material. This circumstance can be traced back to missing cell-to-cell interactions. Recent studies show, that 3D cultures like spheroids or organoids resemble in-vivo cell behavior much more closely than their two-dimensional counterpart \cite{Ranga2014, Elliot2011}. Complementary to that, novel commercial products\cite{insphero, matrigel} allow the efficient and reliable growth of such cultures. Although a lot of research is focused on reactions of spheroid and organoids to medicinal compounds, there seems to be a lack of technical devices that facilitate the production of evidence \citep{Curto2018}. 
In order to address this shortfall, this work shall produce a system that allows the automatic and continuous monitoring of spheroids via an \gls{is} measurement. Usage of this system shall dramatically decrease the effort of cytotoxicity assays and shall pave the way for large scale clinical applications. With the ultimate goal being the acceleration of cancer/drug research, therapy and diagnostics. 

\section{State of the Art}
\label{sec:state_of_the_art}
Curto et al. \cite{Curto2018} proposed a microfluidic system, that enables in-vitro \gls{is}. An \gls{oect} is facilitated as detector and amplifier for the measurement signal. The publication defines the spheroid resistance $R_{sph}$, which is expected to better describe the organoid than \gls{teer}. The organoid is held in place within an nozzle trap. This allows precise positioning and unloading/loading via an inversion of the medium flow direction. However, no continuous flow can be upheld, as this could cause the destruction of the organoid inside the nozzle trap. The authors suggest that this setup is eligible to distinguish between cell types an their viability. However, almost no effect has been reported on variations of spheroid size. 

Gong et al. \cite{Gong2021} followed a simpler approach. The \gls{is} is conveyed through an field's metal electrode structure. The spheroid is not held in place. Rather it is pumped through a closed loop system and is analyzed, every time it passes the electrode array. This setup lends itself to high throughput applications, but is less suitable for configurations where single spheroids shall be monitored continuously.

Honrad et al. \cite{Honrado2021} shows that \gls{is} is even feasible for analytes that comprise of a single cell. The presented microfluidics device includes multiple traps and bypass channels. This construction allows the simultaneous cytometry of multiple cells. 

\section{Research Design}
\label{sec:research_design}
First, a pump control logic shall be designed and implemented, which allows the control of a microfluidics pump and an electrical impedance analyzer. This pump control logic shall allow the manual configuration (i.e. microfluidic flow rate, pressure, on and off time, impedance spectrum, etc.) of the components via an \gls{gui}. This \gls{gui} shall also be used to start an assay. The logic than monitors the microfluidic device and stores any generated data persistently.

The pump control logic shall be tested on existing microdfluidic devices. After successful implementation, a microfluidic chip shall be designed and fabricated, which allows the continuous monitoring of an spheroid/organoid via \gls{is}. Using the pump control logic and the microfluidic device, a well known cytometric assay shall be executed, in order to calibrate the measurement process. The generated data shall be analyzed and metrics shall be defined, which enable easy interpretation of the monitored proceedings. 

\section{Time Plan}
\label{sec:time_plan}

The major project tasks can be seen in the following list. A corresponding gantt chart is displayed in \Cref{fig:projectTimePlan1}.
\begin{enumerate}
\item \textbf{Micro Pump Control:} A control for an micro fluidics pump shall be designed, implemented, tested and documented.
    \begin{enumerate}
        \item \textbf{Research:} Research regarding micro fluidics pump shall be done. 
        \item \textbf{Design:} The software design of the pump control shall be done.
        \item \textbf{Development:} The software for the pump control shall be written.
        \item \textbf{Testing:} The pump control shall be tested.
        \item \textbf{Milestone 1:} The pump control has been successfully tested and its documentation is finished.
    \end{enumerate}
\item \textbf{Impedance Measurement:} A micro fluidics device shall be created, which allows the impedance measurement of organoids.
    \begin{enumerate}
        \item \textbf{Research:} Research regarding organoid impedance measurement shall be done. 
        \item \textbf{Electrode Design:} The electrode for the impedance measurement shall be designed.
        \item \textbf{Design of Measurement:} The impedance measurement principle shall be designed.
        \item \textbf{Simulation of Measurement:} The electrode design shall be modeled, simulated and validated against the measurement design.
        \item \textbf{Fabrication:} The micro fluidics device shall be fabricated.
        \item \textbf{Test Run:} The impedance measurement shall be tested.
        \item \textbf{Test Evaluation:} The test results of the impedance measurement shall be evaluated.
        \item \textbf{Milestone 2:} The impedance measurement has successfully been applied to an organoid.
    \end{enumerate}
\item \textbf{Documentation:} The diploma thesis shall be written.
    \begin{enumerate}
        \item \textbf{Milestone 3:} Diploma thesis has been written and submitted. End of project.
    \end{enumerate}
\end{enumerate}
 
\bibliographystyle{plain}
\bibliography{lib/library}
 
  % Requires \usepackage{pgfgantt} and \usepackage{pdflscape}
 
 \newgeometry{left=15mm,bottom=20mm}
 \begin{landscape}
   \ganttset{calendar week text={W\myWeek{}}}
   \begin{figure}
 \begin{ganttchart}[hgrid, x unit=1.3mm, y unit chart=7.5mm, time slot format=little-endian
]{01.10.2022}{28.2.2023}
    \gantttitlecalendar{year, month = name, week = 40} \\
    \ganttgroup{Micro Pump Control}{01.10.2022}{31.10.2022} \\
        \ganttbar{Research}{01.10.2022}{07.10.2022} \\
        \ganttlinkedbar{Design}{07.10.2022}{14.10.2022} \\
        \ganttlinkedbar{Development}{14.10.2022}{21.10.2022} \\
        \ganttlinkedbar{Testing}{21.10.2022}{28.10.2022} \\
        \ganttlinkedmilestone{Milestone 1}{28.10.2022} \\
    \ganttgroup{Impedance Measurement}{31.10.2022}{31.12.2022} \\
        \ganttbar{Research}{31.10.2022}{7.11.2022} \\
        \ganttbar{Electrode Design}{7.11.2022}{21.11.2022} \ganttlink{elem7}{elem8}\\
        \ganttbar{Design of Measurement}{7.11.2022}{21.11.2022} \ganttlink{elem7}{elem9}\\
        \ganttlinkedbar{Fabrication}{21.11.2022}{28.11.2022} \\
        \ganttlinkedbar{Test Run}{28.11.2022}{30.11.2022} \\
        \ganttlinkedbar{Evaluation of Test Results}{30.11.2022}{31.12.2022} \\
        \ganttlinkedmilestone{Milestone 2}{31.12.2022} \\
    \ganttgroup{Exosome Detection}{01.01.2023}{27.02.2023} \\
        \ganttbar{Research}{01.01.2023}{27.02.2023} \\
        \ganttmilestone{Milestone 3}{27.02.2023} \\
    \ganttgroup{Documentation}{02.10.2022}{27.02.2023} \\
        \ganttmilestone{Milestone 4}{27.02.2023}
\end{ganttchart} 
    \caption[Gantt chart]{The project time plan from october 2022 to february 2023}
    \label{fig:projectTimePlan1} 
   \end{figure} 
 \end{landscape}
\restoregeometry

\newgeometry{left=15mm,bottom=20mm}
 \begin{landscape}
   \ganttset{calendar week text={W\myWeek{}}}
   \begin{figure}
 \begin{ganttchart}[hgrid, x unit=1.3mm, y unit chart=7.5mm, time slot format=little-endian]{01.03.2023}{31.08.2023}
    \gantttitlecalendar{year, month = name, week = 40} \\
    \ganttgroup{Buffer}{01.03.2023}{31.04.2023} \\
\end{ganttchart} 
    \caption[Gantt chart]{The project time plan.}
    \label{fig:projectTimePlan2} 
   \end{figure} 
 \end{landscape}
\restoregeometry
 
 
\end{document}
